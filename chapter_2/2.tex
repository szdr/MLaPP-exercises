\documentclass{jsarticle}
\usepackage{amsmath, amssymb}
\begin{document}
\section*{Exercise 2.1 Probabilities are sensitive to the form of the question that was used to generate the answer}
\subsection*{(a)}
\begin{table}[htbp]
\centering
\begin{tabular}{cc}
C1 & C2 \\ \hline
B & B\\
B & G\\
G & B\\
G & G
\end{tabular}
\end{table}
If the neighbor has any boys, there is only a chance to the first of the three rows in the above table.
In this situation, the probability that one child is a girl is $\frac{2}{3}$.
\subsection*{(b)}
\begin{eqnarray*}
P(Run=C1)\cdot P(C2=G|Run=C1)+P(Run=C2)\cdot P(C1=G|Run=C2) & = & \frac{1}{2}\cdot\frac{1}{2}\\ & = & \frac{1}{2}
\end{eqnarray*}

\section*{Exercise 2.2 Legal reasoning}
\subsection*{(a)}
\begin{eqnarray}
G & = & \begin{cases}
1 & (\rm defendant\ is\ guilty) \\
0 & (\rm otherwise)
\end{cases}\\
B & = & \begin{cases}
1 & (\rm defendant's\ blood\ type\ matches\ one\ at\ the\ scene)\\
0 & (\rm otherwise)
\end{cases}\\
P(G=0|B=1) & = & \frac{P(B=1|G=0)\cdot P(G=0)}{P(B=1)}\\
& = & \frac{P(B=1|G=0)\cdot P(G=0)}{P(B=1|G=1)\cdot P(G=1)+P(B=1|G=0)\cdot P(G=0)}\\
& = & \frac{\frac{1}{1000}\times \frac{799999}{800000}}{1\times \frac{1}{800000} + \frac{1}{1000} \times \frac{799999}{800000}}\\
& \simeq & 0.999
\end{eqnarray}
The probability of the defendant is innocense given that defendant's blood type matches one at the scene is 0.999.
\subsection*{(b)}
The probabilities of guilty of 8000 people are all the same. This fact doesn't show that the defendant is innocense.

\section*{Exercise 2.3 Variance of a sum}
\begin{eqnarray}
V[X+Y] & = & E[\{(X+Y)-E[X+Y]\}^2]\\
& = & E[\{(X+Y)-(E[X]+E[Y])\}^2]\\
& = & E[\{(X-E[X])+(Y-E[Y])\}^2]\\
& = & E[(X-E[X])^2+(Y-E[Y])^2+2(X-E[X])(Y-E[Y])]\\
& = & E[(X-E[X])^2] + E[(Y-E[Y])^2] + 2E[(X-E[X])(Y-E[Y])]\\
& = & V[X] + V[Y] + 2{\rm cov}[X,Y]
\end{eqnarray}

\section*{Exercise 2.4 Bayes rule for medical diagnosis}
\begin{eqnarray}
X & = & \begin{cases}
1 & (\rm I\ have\ the\ disease) \\
0 & (\rm otherwise)
\end{cases}\\
Y & = & \begin{cases}
1 & (\rm testing\ positive)\\
0 & (\rm otherwise)
\end{cases}\\
P(X=1|Y=1) & = & \frac{P(Y=1|X=1)\cdot P(X=1)}{P(Y=1)}\\
& = & \frac{P(Y=1|X=1)\cdot P(X=1)}{P(Y=1|X=1)\cdot P(X=1)+P(Y=1|X=0)\cdot P(X=0)}\\
& = & \frac{\frac{99}{100}\times \frac{1}{10000}}{\frac{99}{100}\times \frac{1}{10000} + \frac{1}{100}\times \frac{9999}{10000}}\\
& = & \frac{99}{99 + 9999}\\
& = & \frac{1}{102}\\
& \simeq & 9.8\times 10^{-3}
\end{eqnarray}

\section*{Exercise 2.5 The Mohty Hall Problem}
\begin{eqnarray}
X & = & \begin{cases}
1  & (\rm the\ prize\ in\ door\ 1) \\
2  & (\rm the\ prize\ in\ door\ 2) \\
3  & (\rm the\ prize\ in\ door\ 3)
\end{cases}\\
Y & = & \begin{cases}
1  & (\rm the\ host\ opens\ door\ 1)\\
2  & (\rm the\ host\ opens\ door\ 2)\\
3  & (\rm the\ host\ opens\ door\ 3)
\end{cases}\\
P(X=1) & = & P(X=2) = P(X=3) = \frac{1}{3}\\
P(Y=3|X=1) & = & \frac{1}{2}\\
P(Y=3|X=2) & = & 1\\
P(Y=3|X=3) & = & 0\\
P(X=1|Y=3) & = & \frac{P(Y=3|X=1)\cdot P(X=1)}{P(Y=3)}\\
& = & \frac{P(Y=3|X=1)\cdot P(X=1)}{\sum_{i=1}^{3}P(Y=3|X=i)\cdot P(X=i)}\\
& = & \frac{\frac{1}{2}\times \frac{1}{3}}{\frac{1}{2}\times \frac{1}{3} + 1 \times \frac{1}{3}+0\times \frac{1}{3}}\\
& = & \frac{1}{3}\\
P(X=3|Y=3) & = & 0\ (\because P(Y=3|X=3)=0)\\
P(X=2|Y=3) & = & 1 - P(X=1|Y=3) - P(X=3|Y=3)\\
& = & \frac{2}{3}
\end{eqnarray}
This result shows the contestant should switch to door 2.

\section*{Exercise 2.6 Conditional Independence}
\subsection*{(a)}
\begin{align}
P(H,e_1,e_2) & = P(e_1,e_2|H)P(H)\\
P(H,e_1,e_2) & = P(H|e_1,e_2)p(e_1,e_2)\\
\therefore p(H|e_1,e_2) & =\frac{P(e_1,e_2|H)P(H)}{P(e_1,e_2)}
\end{align}
Above calculation shows (ii.) is sufficient. 
\subsection*{(b)}
\begin{align}
P(H|e_1,e_2) & = \frac{P(e_1,e_2|H)P(H)}{P(e_1,e_2)}\\
& = \frac{P(e_1|H)P(e_2|H)P(H)}{P(e_1,e_2)} (\because E_1\ {\rm and}\ E_2\ {\rm are\ conditionally\ independent\ given\ H}) \\
P(e_1,e_2) &= \sum_{i=1}^{K}P(e_1,e_2|H=i)P(H=i)\\
&= \sum_{i=1}^{K}P(e_1|H=i)P(e_2|H=i)P(H=i)
\end{align}
Above calculation shows (i.), (ii.) and (iii.) are sufficient.

\section*{Exercise 2.7 Pairwise independence does not imply mutual independence}
(Reference to http://www.cut-the-knot.org/Probability/MutuallyIndependentEvents.shtml)

There are four balls numbered as below.

\begin{align*}
B1 = 110 \\
B2 = 101 \\
B3 = 011 \\
B4 = 000
\end{align*}
For $k=1,2,3$ let $A_k$ be the event of drawing a ball with 1 in the $k$th position.
\begin{align}
P(A_1)=\frac{1}{2},\ P(A_2)=\frac{1}{2},\ P(A_3)=\frac{1}{2}\\
P(A_1,A_2)=\frac{1}{4},\ P(A_1, A_3)=\frac{1}{4},\ P(A_2, A_3)=\frac{1}{4}\\
\therefore P(A_1, A_2)=P(A_1)P(A_2)\\
P(A_1, A_3)=P(A_1)P(A_3)\\
P(A_2,A_3)=P(A_2)P(A_3)
\end{align}
This shows all pairs of variables are pairwise independence.
\begin{align}
P(A_1,A_2,A_3)=0\\
P(A_1)P(A_2)P(A_3)=\frac{1}{8}\\
\therefore P(A_1,A_2,A_3)=P(A_1)P(A_2)P(A_3)
\end{align}
This shows mutual independence doesn't hold.

\section*{Exercise 2.8 Conditional independence iff joint factorizes}
\subsection*{$(\Rightarrow)$}
\begin{align}
g(x,z)=p(x|z), h(y,z)=p(y|z)
\end{align}
\subsection*{$(\Leftarrow)$}
\begin{align}
p(x,y|z) & = g(x,z)h(y,z) \label{2.8}
\end{align}
We calculate the integral of left side and right side.
\begin{align}
\int_{x}p(x,y|z)dx & =\int_{x}g(x,z)h(y,z)dx\\
p(y|z) & = h(y,z)G(x,z) 
\end{align}
We calculate the integral of left side and right side.
\begin{align}
\int_{y}p(y|z)dy & =\int_{y}h(y,z)G(x,z)dy\\
H(y,z)G(x,z) & = 1
\end{align}
We calculate the integral of left side and right side of (\ref{2.8}).
\begin{align}
\int_{y}p(x,y|z)dy & = \int_{y}g(x,z)h(y,z)dy \\
p(x|z) & =g(x,z)H(y,z)
\end{align}
Finally, we get
\begin{align}
p(x,y|z) & = g(x,z)h(y,z)\\
& = \frac{p(x|z)}{H(y,z)}\frac{p(y|z)}{G(x,z)} \\
& = p(x|z)p(y|z)
\end{align}

\section*{Exercise 2.9 Conditional independence}
\subsection*{(a)}
\begin{align}
(X \bot W|Z,Y) & \Leftrightarrow p(X,W|Z,Y) = p(X|Z,Y)p(W|Z,Y) \label{2.9.a.1} \\
(X \bot Y|Z) & \Leftrightarrow p(X,Y|Z) = p(X|Z)p(Y|Z) \label{2.9.a.2} \\
(X\bot Y,W|Z) & \Leftrightarrow p(X,W|Z) = p(X|Z)p(W|Z) \label{2.9.a.3} \\
p(X,W,Z|Y) & = p(X,Y,Z)p(W|Z,Y)\ (\because (\ref{2.9.a.1})) \label{xyzw}\\
p(X,Y,Z) & = p(X,Z)p(Y|Z)\ (\because (\ref{2.9.a.2})) \label{xyz}
\end{align}
Because of (\ref{xyzw}) and (\ref{xyz}), we get
\begin{align}
p(X,Y,Z,W) & = p(X,Z)p(Y|Z)p(W|Z,Y)\\
& = p(X,Z)p(Y,W|Z)
\end{align}
We calculate the integral of left side and right side.
\begin{align}
\int_{Y}p(X,Y,Z,W)dY & = \int_{Y}p(X,Z)p(Y,W|Z)dY\\
p(X,Z,W) & = p(X,Z)p(W|Z)\\
\frac{p(X,W|Z)}{p(Z)} & = \frac{p(X|Z)}{p(Z)}p(W|Z)\\
p(X,W|Z) & = p(X|Z)p(W|Z)
\end{align}
We showed (\ref{2.9.a.1}) $\wedge$ (\ref{2.9.a.2}) $\Rightarrow$ (\ref{2.9.a.3}).
\subsection*{(b)}
Somebody help me!

\section*{Exercise 2.10 Deriving the inverse gamma density}
\begin{align}
p_y(y) & =p_x(x)|\frac{dx}{dy}|\\
 & = \frac{b^a}{\Gamma(a)}x^{a-1}e^{-xb}|-\frac{1}{y^2}|\\
 & = \frac{b^a}{\Gamma(a)}y^{-(a-1)}e^{-\frac{b}{y}}\frac{1}{y^2}\\
 & = \frac{b^a}{\Gamma(a)}y^{-(a+1)}e^{-\frac{b}{y}}\\
 & = IG(y|a,b)
\end{align}

\section*{Exercise 2.11 Normalization constant for a 1D Gaussian}
\begin{align}
Z^2 & = \int_{0}^{2\pi}\int_{0}^{\infty}r\exp(-\frac{r^2}{2\sigma^2})drd\theta\\
& = \int_{0}^{2\pi}d\theta\int_{0}^{\infty}r\exp(-\frac{r^2}{2\sigma^2})dr\\
& = 2\pi \int_{0}^{\infty}r\exp(-\frac{r^2}{2\sigma^2})dr\\
& = 2\pi \left[-\sigma^2\exp(-\frac{r^2}{2\sigma^2})\right]_{0}^{\infty}\\
& = 2\pi\sigma^2\\
\therefore Z & = \sqrt{2\pi \sigma^2}
\end{align}

\section*{Exercise 2.12 Expressing mutual information in terms of entropies}
\begin{align}
I(X,Y) & = \sum_x\sum_yp(x,y)\log\frac{p(x,y)}{p(x)p(y)}\\
& = \sum_x\sum_yp(x,y)\log\frac{p(x|y)p(y)}{p(x)p(y)}\\
& = \sum_x\sum_yp(x,y)(-\log p(x)+\log p(x|y))\\
& = -\sum_x\sum_yp(x,y)\log p(x) + \sum_x\sum_yp(x,y)\log p(x,y)\\
& = -\sum_xp(x)\log p(x) + \sum_x\sum_yp(x,y)\log p(x|y)\\
& = H(x) + \sum_x\sum_yp(x,y)\log p(x|y)\\
& = H(X) + \sum_x\sum_yp(y)p(x|y)\log p(x|y)\\
& = H(X) + \sum_y p(y)\sum_xp(x|y)\log p(x|y)\\
& = H(X) - \sum_y p(y) H(X|Y=y)\\
& = H(X) - H(X|Y)
\end{align}
$I(X,Y)=H(Y)-H(X|Y)$ can be shown like as below.
\end{document}