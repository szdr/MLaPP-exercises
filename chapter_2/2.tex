\documentclass{jsarticle}
\usepackage{amsmath}
\begin{document}
\section*{Exercise 2.1 Probabilities are sensitive to the form of the question that was used to generate the answer}
\subsection*{(a)}
\begin{table}[htbp]
\centering
\begin{tabular}{cc}
C1 & C2 \\ \hline
B & B\\
B & G\\
G & B\\
G & G
\end{tabular}
\end{table}
If the neighbor has any boys, there is only a chance to the first of the three rows in the above table.
In this situation, the probability that one child is a girl is $\frac{2}{3}$.
\subsection*{(b)}
\begin{eqnarray*}
P(Run=C1)\cdot P(C2=G|Run=C1)+P(Run=C2)\cdot P(C1=G|Run=C2) & = & \frac{1}{2}\cdot\frac{1}{2}\\ & = & \frac{1}{2}
\end{eqnarray*}

\section*{Exercise 2.2 Legal reasoning}
\subsection*{(a)}
\begin{eqnarray}
G & = & \begin{cases}
1 & (\rm defendant\ is\ guilty) \\
0 & (\rm otherwise)
\end{cases}\\
B & = & \begin{cases}
1 & (\rm defendant's\ blood\ type\ matches\ one\ at\ the\ scene)\\
0 & (\rm otherwise)
\end{cases}\\
P(G=0|B=1) & = & \frac{P(B=1|G=0)\cdot P(G=0)}{P(B=1)}\\
& = & \frac{P(B=1|G=0)\cdot P(G=0)}{P(B=1|G=1)\cdot P(G=1)+P(B=1|G=0)\cdot P(G=0)}\\
& = & \frac{\frac{1}{1000}\times \frac{799999}{800000}}{1\times \frac{1}{800000} + \frac{1}{1000} \times \frac{799999}{800000}}\\
& \simeq & 0.999
\end{eqnarray}
The probability of the defendant is innocense given that defendant's blood type matches one at the scene is 0.999.
\subsection*{(b)}
The probabilities of guilty of 8000 people are all the same. This fact doesn't show that the defendant is innocense.
\end{document}