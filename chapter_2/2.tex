\documentclass{jsarticle}
\usepackage{amsmath, amssymb}
\begin{document}
\section*{Exercise 2.1 Probabilities are sensitive to the form of the question that was used to generate the answer}
\subsection*{(a)}
\begin{table}[htbp]
\centering
\begin{tabular}{cc}
C1 & C2 \\ \hline
B & B\\
B & G\\
G & B\\
G & G
\end{tabular}
\end{table}
If the neighbor has any boys, there is only a chance to the first of the three rows in the above table.
In this situation, the probability that one child is a girl is $\frac{2}{3}$.
\subsection*{(b)}
\begin{eqnarray*}
P(Run=C1)\cdot P(C2=G|Run=C1)+P(Run=C2)\cdot P(C1=G|Run=C2) & = & \frac{1}{2}\cdot\frac{1}{2}\\ & = & \frac{1}{2}
\end{eqnarray*}

\section*{Exercise 2.2 Legal reasoning}
\subsection*{(a)}
\begin{eqnarray}
G & = & \begin{cases}
1 & (\rm defendant\ is\ guilty) \\
0 & (\rm otherwise)
\end{cases}\\
B & = & \begin{cases}
1 & (\rm defendant's\ blood\ type\ matches\ one\ at\ the\ scene)\\
0 & (\rm otherwise)
\end{cases}\\
P(G=0|B=1) & = & \frac{P(B=1|G=0)\cdot P(G=0)}{P(B=1)}\\
& = & \frac{P(B=1|G=0)\cdot P(G=0)}{P(B=1|G=1)\cdot P(G=1)+P(B=1|G=0)\cdot P(G=0)}\\
& = & \frac{\frac{1}{1000}\times \frac{799999}{800000}}{1\times \frac{1}{800000} + \frac{1}{1000} \times \frac{799999}{800000}}\\
& \simeq & 0.999
\end{eqnarray}
The probability of the defendant is innocense given that defendant's blood type matches one at the scene is 0.999.
\subsection*{(b)}
The probabilities of guilty of 8000 people are all the same. This fact doesn't show that the defendant is innocense.

\section*{Exercise 2.3 Variance of a sum}
\begin{eqnarray}
V[X+Y] & = & E[\{(X+Y)-E[X+Y]\}^2]\\
& = & E[\{(X+Y)-(E[X]+E[Y])\}^2]\\
& = & E[\{(X-E[X])+(Y-E[Y])\}^2]\\
& = & E[(X-E[X])^2+(Y-E[Y])^2+2(X-E[X])(Y-E[Y])]\\
& = & E[(X-E[X])^2] + E[(Y-E[Y])^2] + 2E[(X-E[X])(Y-E[Y])]\\
& = & V[X] + V[Y] + 2{\rm cov}[X,Y]
\end{eqnarray}

\section*{Exercise 2.4 Bayes rule for medical diagnosis}
\begin{eqnarray}
X & = & \begin{cases}
1 & (\rm I\ have\ the\ disease) \\
0 & (\rm otherwise)
\end{cases}\\
Y & = & \begin{cases}
1 & (\rm testing\ positive)\\
0 & (\rm otherwise)
\end{cases}\\
P(X=1|Y=1) & = & \frac{P(Y=1|X=1)\cdot P(X=1)}{P(Y=1)}\\
& = & \frac{P(Y=1|X=1)\cdot P(X=1)}{P(Y=1|X=1)\cdot P(X=1)+P(Y=1|X=0)\cdot P(X=0)}\\
& = & \frac{\frac{99}{100}\times \frac{1}{10000}}{\frac{99}{100}\times \frac{1}{10000} + \frac{1}{100}\times \frac{9999}{10000}}\\
& = & \frac{99}{99 + 9999}\\
& = & \frac{1}{102}\\
& \simeq & 9.8\times 10^{-3}
\end{eqnarray}

\section*{Exercise 2.5 The Mohty Hall Problem}
\begin{eqnarray}
X & = & \begin{cases}
1  & (\rm the\ prize\ in\ door\ 1) \\
2  & (\rm the\ prize\ in\ door\ 2) \\
3  & (\rm the\ prize\ in\ door\ 3)
\end{cases}\\
Y & = & \begin{cases}
1  & (\rm the\ host\ opens\ door\ 1)\\
2  & (\rm the\ host\ opens\ door\ 2)\\
3  & (\rm the\ host\ opens\ door\ 3)
\end{cases}\\
P(X=1) & = & P(X=2) = P(X=3) = \frac{1}{3}\\
P(Y=3|X=1) & = & \frac{1}{2}\\
P(Y=3|X=2) & = & 1\\
P(Y=3|X=3) & = & 0\\
P(X=1|Y=3) & = & \frac{P(Y=3|X=1)\cdot P(X=1)}{P(Y=3)}\\
& = & \frac{P(Y=3|X=1)\cdot P(X=1)}{\sum_{i=1}^{3}P(Y=3|X=i)\cdot P(X=i)}\\
& = & \frac{\frac{1}{2}\times \frac{1}{3}}{\frac{1}{2}\times \frac{1}{3} + 1 \times \frac{1}{3}+0\times \frac{1}{3}}\\
& = & \frac{1}{3}\\
P(X=3|Y=3) & = & 0\ (\because P(Y=3|X=3)=0)\\
P(X=2|Y=3) & = & 1 - P(X=1|Y=3) - P(X=3|Y=3)\\
& = & \frac{2}{3}
\end{eqnarray}
This result shows the contestant should switch to door 2.
\end{document}